\chapter{Introduction}

\section{Background}
\newacronym{ugc}{UGC}{User Generated Content}
\newacronym{it}{IT}{Information Technology}
Social media like Facebook and Twitter are gradually changing the world, and becoming a new source of knowledge. Users generate the data with a feeling of rewarding, since they can get recognition and interaction from other users\cite{krumm2008user}; but for practitioners in \gls{it} industry, \gls{ugc} means a new, inexpensive and fast way to obtain data that is barely impossible in the past.

LinkedIn.com, the world's largest profession network\footnote{According to their official website: http://www.linkedin.com/about-us}, contains a large amount of hidden career and country base industry information, but yet waiting to be discovered. Unlike Facebook and Twitter, the study of LinkedIn.com is not getting as much attentions as it should be. 

\section{Motivation}
The lack of study of LinkedIn.com provides an opportunity. This project use LinkedIn Ireland as the study subject, and aims to develop a reusable, queryable knowledge model for LinkedIn public profiles. The importance of the project can be described in several aspects:

Firstly, Getting fully insights about Ireland industry distribution, personal skills and professionals' education background are always important for a number of people. And it can be easily to scale to LinkedIn worldwide.

Secondly, it's a good complement for government statistics about industries and professionals. The effectiveness and timeliness of \acrshort{ugc} can always guarantee we are getting the first hand data.

Thirdly, at the end of this project, a online public dataset will be provided so that everyone who interest in Ireland facts can query the public endpoint to get information.

\newacronym{hr}{HR}{Human Resources}
Finally, user interface can built on top of the knowledge model and the dataset to support complex queries and answer questions. Government, practitioners in \gls{hr}, and job seekers are possible target audience.

%Firstly, government can use \acs{ugc} in LinkedIn as an aid of decision making. As most of the time, there's a delay between industries and government statistics, the effectiveness and timeliness of \acs{ugc} can be a good approach to know the things happen now.
%
%\newacronym{hr}{HR}{Human Resources}
%Secondly, based on education background and people's skills, practioners of \gls{hr} can easily target the right graduates for vacacies.
%
%Thirdly, if students and job seekers know about the number of people and the number of companies in different areas, they can make more wise decision in job searching and skill learning.

\section{Research Question}

\section{Outline}
\paragraph{}
Chapter 2 discusses the state of the art in Data extraction and Knowledge modelling.
\paragraph{}
Chapter 3 covers system requirements, system architecture and design decisions.
\paragraph{}
Chapter 4 provides details about the implementations, including profile crawling strategy, data extraction, data normalisation and missing fields inference.
\paragraph{}
Chapter 5 evaluates the extracted data accuracy, LinkedIn profile completeness, data linkage and data fitness.
\paragraph{}
Chapter 6 concludes the results and contributions, and future works will be discussed in this chapter as well.
